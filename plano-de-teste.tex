\documentclass[12pt,a4paper]{article}

\usepackage[utf8]{inputenc}
\usepackage[T1]{fontenc}

\usepackage[export]{adjustbox}
\usepackage{graphicx}
\usepackage{float}
\setkeys{Gin}{width=0.8\textwidth}

\usepackage{caption}
\captionsetup{font=small}

\usepackage[hidelinks]{hyperref}
\usepackage{longtable}
\usepackage{geometry}
\usepackage[brazil]{babel}

\setcounter{section}{1}

\geometry{margin=2.5cm}

% ----------------------------------------------------------------------------------------

% TRADUCAO DE COMANDOS %

% ----------------------------------------------------------------------------------------

\renewcommand{\chapterautorefname}{Capítulo}
\renewcommand{\sectionautorefname}{Seção}
\renewcommand{\subsectionautorefname}{Subseção}
\renewcommand{\subsubsectionautorefname}{Subsubseção}
\renewcommand{\figureautorefname}{Figura}
\renewcommand{\tableautorefname}{Tabela}

% ----------------------------------------------------------------------------------------

% INTRODUCAO %

% ----------------------------------------------------------------------------------------

\begin{document}

\section*{\textbf{Plano de Teste do Sistema Figma}}
\vspace{0.2cm}
\small Última atualização: \today

% ----------------------------------

\section*{\large Dados do Projeto}
\label{sec:plano_teste}

\begin{itemize}
    \item \textbf{Nome do sistema:} Figma
    \item \textbf{Ambiente:} Web (Desktop)
    \item \textbf{Responsável:} Ana Flavia de Castro Segadilha da Silva
    \item \textbf{Objetivo do Plano:} Testar a funcionalidade das principais funções do Figma, garantindo que todos os casos testados funcionem conforme especificado neste documento.
\end{itemize}

% ----------------------------------

\subsection{Escopo}
\label{subsec:escopo_teste}
O escopo deste plano de teste cobre ações feitas no Figma Design, como:
\begin{itemize}
    \item Criação e edição de arquivos;
    \item Manipulação de \textit{frames} e objetos;
    \item Prototipação;
    \item Comentários;
    \item Movimentação e exclusão de arquivos;
    \item Compartilhamento de links.
\end{itemize}

% ----------------------------------

\subsubsection*{Fora do Escopo}
\label{subsec:foraescopo_teste}
\begin{itemize}
    \item Uso de \textit{plugins} ou integrações externas;
    \item Funcionalidades de exportação;
    \item Outras ferramentas do Figma.
\end{itemize}

% ----------------------------------

\subsection{Estratégia de Teste}
\label{subsec:estrategia_teste}
Os testes serão executados manualmente, utilizando casos de teste e declarando pré-condições, passos a passo e resultados esperados definidos para cada um deles. 

% ----------------------------------------------------------------------------------------

% CASOS DE TESTE %

% ----------------------------------------------------------------------------------------

\section{Casos de Teste}
\label{sec:casos_teste}

Cada caso de teste possui pré-condições, resultados esperados e passos de execução detalhados. O plano contempla a criação, edição, prototipação e compartilhamento de arquivos de design, garantindo que o sistema se comporte conforme especificado.

% ----------------------------------

\subsection{Casos de Teste}
\label{subsec:resumo_casos}

\begin{longtable}{|p{1cm}|p{5cm}|p{4cm}|p{5cm}|}
\hline
\textbf{ID} & \textbf{Objetivo} & \textbf{Pré-condições} & \textbf{Resultado Esperado} \\ \hline
CT1 & Criar um novo arquivo de design & Usuário autenticado, tela inicial carregada & Novo arquivo visível na lista de arquivos \\ \hline
CT2 & Criar um novo \textit{frame} & Usuário autenticado, Arquivo de design existente, Arquivo de design aberto & \textit{Frame} criado corretamente \\ \hline
CT3 & Prototipar interações & Usuário autenticado, Arquivo de design existente, Arquivo de design aberto, Ao menos dois \textit{frames} & Interações aplicadas corretamente \\ \hline
CT4 & Adicionar objeto no \textit{frame} & Usuário autenticado, Arquivo de design existente, Arquivo de design aberto, Ao menos um \textit{frame} criado & Objeto aparece no \textit{frame} \\ \hline
CT5 & Editar objeto no arquivo & Usuário autenticado, Arquivo de design existente, Arquivo de design aberto, Ao menos um objeto existente & Objeto alterado conforme propriedades escolhidas \\ \hline
CT6 & Comentar em algum \textit{frame} ou objeto & Usuário autenticado, Arquivo de design existente, Arquivo de design aberto, Ao menos um \textit{frame} ou objeto existente selecionado & Comentário visível para todos usuários \\ \hline
CT7 & Desenhar no arquivo & Usuário autenticado, Arquivo de design existente, Arquivo de design aberto & Desenho aparece corretamente no \textit{canvas} \\ \hline
CT8 & Compartilhar link para visualização & Usuário autenticado, Arquivo de design existente, Arquivo de design aberto & Link ou permissão de visualização funcionando \\ \hline
CT9 & Mover arquivo para uma equipe & Usuário autenticado, Arquivo de design existente, Arquivo de design aberto & Arquivo movido para pasta da equipe correta \\ \hline
CT10 & Deletar arquivo & Usuário autenticado, Arquivo de design existente, Arquivo de design aberto & Arquivo enviado para lixeira e deletado permanentemente \\ \hline
\end{longtable}

% ----------------------------------

\subsection{Casos de Teste Detalhados}
\label{sec:casos_detalhados}

% -------------------------------
\subsubsection{CT1: Criar um novo arquivo de design}
\label{subsubsec:ct1}

\textbf{Pré-condições}

\begin{itemize}
    \item Usuário autenticado no Figma;
    \item Tela inicial carregada.
\end{itemize}

\textbf{Passos a passo}

\begin{enumerate}
    \item Clique no botão \textit{Design} na tela inicial (\autoref{fig:tela_principal});
    \item Aguarde o carregamento do editor, e o arquivo será criado. \\
\end{enumerate}


\textbf{Resultado esperado:} \\
\vspace{0.2cm}
O Figma cria um novo arquivo de design na lista de arquivos do usuário.

\begin{figure}[H]
    \centering
    \includegraphics{imagens/tela_principal.png}
    \caption{Tela inicial do Figma}
    \label{fig:tela_principal}
\end{figure}

% -------------------------------

\subsubsection{CT2: Criar um novo \textit{frame}}
\label{subsubsec:ct2}

\textbf{Pré-condições}
\begin{itemize}
    \item Usuário autenticado no Figma;
    \item Arquivo de design existe;
    \item Arquivo de design aberto.
\end{itemize}

\textbf{Passos de execução}
\begin{enumerate}
    \item Clique no ícone de \textit{Frame};
    \item Escolha um tamanho predefinido (ex.: iPhone 16);
    \item Confirme a criação do \textit{frame}. \\
\end{enumerate}

\textbf{Resultado esperado} \\
\vspace{0.2cm}
O \textit{frame} é criado corretamente e aparece no painel de camadas.

\begin{figure}[H]
    \centering
    \includegraphics{imagens/tela_design_frame.png}
    \caption{Criação de um \textit{frame}}
    \label{fig:tela_design_frame}
\end{figure}

% -------------------------------

\subsubsection{CT3: Prototipação de interações}
\label{subsubsec:ct3}


\textbf{Pré-condições}
\begin{itemize}
    \item Usuário autenticado no Figma;
    \item Arquivo de design existente;
    \item Arquivo de design aberto;
    \item Ao menos dois \textit{frames} criados.
\end{itemize}

\textbf{Passos de execução}
\begin{enumerate}
    \item Abra a aba \textit{Protótipo};
    \item Clique no botão de interações;
    \item Selecione gatilho, ação e destino;
    \item Configure animação e comportamento no painel direito. \\
\end{enumerate}

\textbf{Resultado esperado} \\
\vspace{0.2cm}
As interações são aplicadas corretamente e podem ser visualizadas no modo de pré-visualização do protótipo.

\begin{figure}[H]
    \centering
    \includegraphics{imagens/tela_design_prototipo.png}
    \caption{Criação de uma interação}
    \label{fig:tela_design_prototipo}
\end{figure}

% -------------------------------

\subsubsection{CT4: Adicionar objeto no \textit{frame}}
\label{subsubsec:ct4}

\textbf{Pré-condições}
\begin{itemize}
    \item Usuário autenticado no Figma;
    \item Arquivo de design existente;
    \item Arquivo de design aberto;
    \item Ao menos um \textit{frame} existente.
\end{itemize}

\textbf{Passos de execução}
\begin{enumerate}
    \item Clique no ícone de objeto;
    \item Escolha o tipo de objeto (ex.: retângulo);
    \item Confirme a criação do objeto no \textit{frame}. \\
\end{enumerate}

\textbf{Resultado esperado} \\
\vspace{0.2cm}
Objeto criado corretamente e visível no \textit{frame}.

\begin{figure}[H]
    \centering
    \includegraphics{imagens/tela_design_objeto.png}
    \caption{Criação de um objeto}
    \label{fig:tela_design_objeto}
\end{figure}

% -------------------------------

\subsubsection{CT5: Editar objeto no arquivo}
\label{subsubsec:ct5}

\textbf{Pré-condições}
\begin{itemize}
    \item Usuário autenticado no Figma;
    \item Arquivo de design existente;
    \item Arquivo de design aberto;
    \item Ao menos um objeto existente.
\end{itemize}

\textbf{Passos de execução}
\begin{enumerate}
    \item Selecione o objeto;
    \item Modifique propriedades (cor, tamanho, etc.);
    \item Confirme alterações. \\
\end{enumerate}

\textbf{Resultado esperado} \\
\vspace{0.2cm}
Objeto alterado conforme as propriedades escolhidas.

\begin{figure}[H]
    \centering
    \includegraphics{imagens/tela_design_editar.png}
    \caption{Edição de um objeto}
    \label{fig:tela_design_editar}
\end{figure}

% -------------------------------

\subsubsection{CT6: Comentar em algum objeto}
\label{subsubsec:ct6}

\textbf{Pré-condições}
\begin{itemize}
    \item Usuário autenticado no Figma;
    \item Arquivo de design existente;
    \item Arquivo de design aberto;
    \item Ao menos um \textit{frame} ou objeto existente.
\end{itemize}


\textbf{Passos de execução}
\begin{enumerate}
    \item Clique no ícone de comentário;
    \item Selecione o objeto;
    \item Escreva e adicione o comentário. \\
\end{enumerate}

\textbf{Resultado esperado} \\
\vspace{0.2cm}
Comentário visível para todos usuários com acesso ao arquivo.

\begin{figure}[H]
    \centering
    \includegraphics{imagens/tela_design_comentario.png}
    \caption{Criação de um \textit{comentário}}
    \label{fig:tela_design_comentario}
\end{figure}

% -------------------------------

\subsubsection{CT7: Desenhar no arquivo}
\label{subsubsec:ct7}

\textbf{Pré-condições}
\begin{itemize}
    \item Usuário autenticado no Figma;
    \item Arquivo de design existente;
    \item Arquivo de design aberto.
\end{itemize}


\textbf{Passos de execução}
\begin{enumerate}
    \item Clique na ferramenta de desenho;
    \item Desenhe no \textit{canvas}. \\
\end{enumerate}

\textbf{Resultado esperado} \\
\vspace{0.2cm}
Desenhos aparecem corretamente no \textit{canvas} como vetores.

\begin{figure}[H]
    \centering
    \includegraphics{imagens/tela_design_desenho.png}
    \caption{Desenho em forma de vetor}
    \label{fig:tela_design_desenho}
\end{figure}

% -------------------------------

\subsubsection{CT8: Compartilhar link para visualização}
\label{subsubsec:ct8}

\textbf{Pré-condições}
\begin{itemize}
    \item Usuário autenticado no Figma;
    \item Arquivo de design existente;
    \item Arquivo de design aberto.
\end{itemize}

\textbf{Passos de execução}
\begin{enumerate}
    \item Clique em “Compartilhar”;
    \item Escreva o e-mail para quem deseja compartilhar;
    \item Escolha as permissões de acesso (\autoref{fig:tela_design_opcoes});
    \item Aperte no botão "Convidar". \\
\end{enumerate}

\textbf{Resultado esperado} \\
\vspace{0.2cm}
Arquivo compartilhado com sucesso, com as permissões corretas aplicadas e o novo usuário aparece na seção "quem tem acesso".

\begin{figure}[H]
    \centering
    \includegraphics{imagens/tela_design_opcoes.png}
    \caption{Compartilhamento de Arquivos}
    \label{fig:tela_design_opcoes}
\end{figure}

% -------------------------------

\subsubsection{CT9: Mover arquivo para uma equipe}
\label{subsubsec:ct9}


\textbf{Pré-condições}
\begin{itemize}
    \item Usuário autenticado no Figma;
    \item Arquivo de design existente;
    \item Arquivo de design aberto.
\end{itemize}


\textbf{Passos de execução}
\begin{enumerate}
    \item Abra menu de compartilhamento;
    \item Clique em “Mover Arquivo” (\autoref{fig:tela_design_opcoes});
    \item Selecione equipe de destino (\autoref{fig:tela_design_mover}). \\
\end{enumerate}

\textbf{Resultado esperado} \\
\vspace{0.2cm}
Arquivo transferido para equipe e visível para os membros.

\begin{figure}[H]
    \centering
    \includegraphics{imagens/tela_design_mover.png}
    \caption{Menu de compartilhamento}
    \label{fig:tela_design_mover}
\end{figure}

% -------------------------------

\subsubsection{CT10: Deletar arquivo}
\label{subsubsec:ct10}

\textbf{Pré-condições}
\begin{itemize}
    \item Usuário autenticado no Figma;
    \item Arquivo de design existente;
    \item Arquivo de design aberto.
\end{itemize}

\textbf{Passos de execução}
\begin{enumerate}
    \item Selecione arquivo na tela inicial (\autoref{fig:tela_principal}) e clique em “Mover para lixeira”;
    \item Abra a aba "Lixeira" e selecione o arquivo novamente (\autoref{fig:tela_design_excluir_permanentemente}) para excluir permanentemente. \\
\end{enumerate}

\textbf{Resultado esperado} \\
\vspace{0.2cm}
Arquivo removido e não pode ser recuperado.

\begin{figure}[H]
    \centering
    \includegraphics{imagens/tela_design_excluir_permanentemente.png}
    \caption{Menu da Lixeira}
    \label{fig:tela_design_excluir_permanentemente}
\end{figure}

% ----------------------------------

\end{document}
