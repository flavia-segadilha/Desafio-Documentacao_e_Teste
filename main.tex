\documentclass[12pt,a4paper]{article}

\usepackage[utf8]{inputenc}
\usepackage[T1]{fontenc}

\usepackage[export]{adjustbox}
\usepackage{graphicx}
\usepackage{float}
\setkeys{Gin}{width=0.8\textwidth}

\usepackage{caption}
\captionsetup{font=small}

\usepackage[hidelinks]{hyperref}
\usepackage{array}
\usepackage{longtable}
\usepackage{titlesec}
\usepackage{geometry}
\usepackage[brazil]{babel}

\geometry{margin=2.5cm}

% ----------------------------------------------------------------------------------------

% TRADUCAO DE COMANDOS %

% ----------------------------------------------------------------------------------------

\renewcommand{\chapterautorefname}{Capítulo}
\renewcommand{\sectionautorefname}{Seção}
\renewcommand{\subsectionautorefname}{Subseção}
\renewcommand{\subsubsectionautorefname}{Subsubseção}
\renewcommand{\figureautorefname}{Figura}
\renewcommand{\tableautorefname}{Tabela}

% ----------------------------------------------------------------------------------------

% PAGINA INICIAL %

% ----------------------------------------------------------------------------------------

\begin{titlepage}
    \title{
        \centering
        \vspace*{5cm}
        \textbf{Manual do Usuário  Figma} \\
        \vspace{0.5cm}
        \large Desafio Documentação --- Processo Seletivo ADA
        \vspace{1.2cm}
    }
    
    \author{Ana Flavia de Castro Segadilha da Silva}
    \date{\today}
    
    \vfill
\end{titlepage}

\begin{document}

\maketitle
\newpage

% ----------------------------------------------------------------------------------------

% SUMARIO %

% ----------------------------------------------------------------------------------------

\tableofcontents
\newpage

% ----------------------------------------------------------------------------------------

% LISTA DE FIGURAS %

% ----------------------------------------------------------------------------------------

\listoffigures
\newpage

% ----------------------------------------------------------------------------------------

% INTRODUCAO %

% ----------------------------------------------------------------------------------------
\section{Introdução ao Sistema}
\label{sec:introducao}

\subsection{O que é o Figma}
\label{subsec:figma}
O Figma é uma plataforma web colaborativa para design e prototipação. Mais conhecido por sua aplicação na criação de interfaces, é também possível produzir integralmente materiais gráficos como slides, templates e outros recursos para diversas utilidades. \\


Atualmente, o Figma possui 6 principais modos de uso: Design, onde é possível desenvolver designs de interfaces e a prototipação destes, FigJam, uma \textit{whiteboard} interativa e colaborativa, Slides, a plataforma de criação de slides do Figma, Buzz, em que são feitos designs para recursos digitais como publicações em redes sociais, Site, o recurso de design e publicação de sites no Figma, e Make, a ferramenta de IA que pode gerar e transformar os designs criados em protótipos funcionais com implementação em código editável.

\vspace{1cm}

% ----------------------------------

\subsection{Qual problema resolve}
\label{subsec:justificativa}
A proposta do Figma como sistema é tranformar o fluxo de design para algo mais colaborativo. Em outros sistemas, o design e prototipação são desenvolvidos localmente e de forma individual, o que pode causar demora na comunicação e dificuldade no compartilhamento de arquivos atualizados. \\

Como o Figma é uma plataforma online e colaborativa que possui ferramentas de design, prototipação e codificação em aoenas um sistema, é possível que toda a equipe tenha uma visão do trabalho em tempo real e feedbacks mais rápidos do que deve ser feito e modificado, o que diminui o risco de desententimentos internos e externos entre equipes e aumento na produtividade geral.

\vspace{1cm}

% ----------------------------------

\subsection{Público-alvo}
\label{subsec:publico}
O público-alvo do Figma é composto principalmente por:
\begin{itemize}
    \item Designers de interface (UI/UX), que criam interfaces e protótipos para sistemas;
    \item Times de desenvolvimento, que podem utilizar as ferramentas para validar ideias, acompanhar o desenvolvimento do sistema e colaborar entre si;
    \item Profissionais de marketing e branding, que usam das vastas ferramentas para criar e publicar suas ideias;
    \item Educadores e estudantes, que utilizam o Figma como ferramenta de apoio no aprendizado em design;
    \item Desenvolvedores, que se beneficiam de uma plataforma simplificada e versátil para a implementação visual dos sistemas;
\end{itemize}

\newpage
% ----------------------------------------------------------------------------------------

% ACESSO AO SISTEMA %

% ----------------------------------------------------------------------------------------

\section{Acesso ao Sistema}
\label{sec:acesso}

Abaixo está a tela inicial do Figma:
\begin{figure}[H]
    \centering
    \includegraphics{imagens/tela_inicial.png}
    \caption{Tela inicial do Figma}
    \label{fig:tela_inicial}
\end{figure}

Em caso de primeiro uso em um dispositivo, é preciso fazer o Cadastro (\autoref{subsec:cadastro}) ou Login (\autoref{subsec:login}) de uma conta para acessá-lo. Nesta seção, será demonstrado as diversas formas de como este acesso pode ser realizado.

\vspace{1cm}

% ----------------------------------

\subsection{Cadastro de conta}
\label{subsec:cadastro}

\subsubsection{Guia rápido}
\label{subsec:signup_rapido}

\begin{enumerate}
    \item Acesse o site \url{https://www.figma.com};
    \item Clique em \textit{Get started for Free} (\autoref{fig:tela_inicial_signup});
    \item Informe seu e-mail e senha ou utilize cadastro via Google (\autoref{fig:tela_signup});
    \item Confirme o e-mail enviado pelo Figma.
\end{enumerate}

\vspace{0.2cm}

\subsubsection{Passo a passo}
\label{subsubsec:signup_passoapasso}

Para fazer uma nova conta no Figma, é preciso apertar o botão \textit{Get started for Free}, indicado abaixo.

\begin{figure}[H]
    \centering
    \includegraphics{imagens/tela_inicial_signup.png}
    \caption{Botão pode ser encontrado no canto superior direito da tela}
    \label{fig:tela_inicial_signup}
\end{figure}

\vspace{0.2cm}

Após apertar no botão de cadastro, uma janela pop-up aparecerá para escolher entre o cadastro de e-mail e cadastro google (\autoref{fig:tela_signup})

\begin{figure}[H]
    \centering
    \includegraphics{imagens/tela_signup.png}
    \caption{Opções de cadastro}
    \label{fig:tela_signup}
\end{figure}

\vspace{0.2cm}

Para criar uma conta usando o seu e-mail, digite no campo indicado o e-mail que será usado no cadastro
\begin{figure}[H]
    \centering
    \includegraphics{imagens/tela_signup_email.png}
    \caption{Após digitar seu e-mail, aperte o botão \textit{Continue with e-mail}}
    \label{fig:tela_signup_email}
\end{figure}

\vspace{0.2cm}

Após isso, será solicitado que você escolha uma senha e, novamente, aperte no botão para continuar

\begin{figure}[H]
    \centering
    \includegraphics{imagens/tela_signup_senha.png}
    \caption{Aperte o botão \textit{Create account}}
    \label{fig:tela_signup_senha}
\end{figure}

\vspace{0.2cm}

Após isso, será enviado um e-mail para seu correio eletônico com as informações necessárias para continuar seu cadastro. É preciso que você confirme e siga os passos dados no e-mail.

\begin{figure}[H]
    \centering
    \includegraphics{imagens/tela_email.png}
    \caption{Aperte o botão \textit{Log in to Figma}}
    \label{fig:tela_email}
\end{figure}

\vspace{0.2cm}

% ----------------------------------

\subsection{Fazer login}
\label{subsec:login}

\subsubsection{Guia rápido}
\label{subsec:login_rapido}

\begin{enumerate}
    \item Acesse o site \url{https://www.figma.com};
    \item Clique em \textit{Log in} (\autoref{fig:tela_inicial_login});
    \item Informe seu e-mail e senha ou utilize login via Google (\autoref{fig:tela_login});
\end{enumerate}

\vspace{0.2cm}

\subsubsection{Passo a passo}
\label{subsubsec:login_passoapasso}

Primeiramente, aperte o botão \textit{Log in} indicado na imagem abaixo (\autoref{fig:tela_inicial_login}):
\begin{figure}[H]
    \centering
    \includegraphics{imagens/tela_inicial_login.png}
    \caption{Aperte o botão \textit{Log in}}
    \label{fig:tela_inicial_login}
\end{figure}

Preencha os dados corretos e aperte novamente no botão \textit{Log in}.

\begin{figure}[H]
    \centering
    \includegraphics{imagens/tela_login.png}
    \caption{Aperte o botão \textit{Log in}}
    \label{fig:tela_login}
\end{figure}

% ----------------------------------

\subsection{Pré-requisitos de acesso}
\label{subsec:requisitos}

Para acessar o Figma, é necessário:
\begin{enumerate}
    \item Estar em ambiente \href{https://pt.wikipedia.org/wiki/Computador_de_escrit%C3%B3rio}{\textit{desktop}} e
    \item Ter uma conta cadastrada e verificada;
\end{enumerate}

É possível visualizar projetos e protótipos em ambientes \href{https://pt.wikipedia.org/wiki/Telefone_celular}{\textit{mobile}}, mas a edição é exclusivamente otimizada para \href{https://pt.wikipedia.org/wiki/Computador_de_escrit%C3%B3rio}{\textit{desktop}}.

% ----------------------------------------------------------------------------------------

% INTERFACE %

% ----------------------------------------------------------------------------------------

\section{Visão Geral da Interface}
\label{sec:interface_geral}

A tela inicial do Figma pode ser vista abaixo:

\begin{figure}[H]
    \centering
    \includegraphics{imagens/tela_principal.png}
    \caption{Tela inicial do sistema}
    \label{fig:tela_principal}
\end{figure}

Será demonstrado abaixo na \autoref{subsec:interface_principais_areas} todas as funcionalidades desta tela.

\subsection{Principais áreas da tela}
\label{subsec:interface_principais_areas}

Pode-se separar a tela inicial em 4 blocos principais vistos na imagem abaixo (\autoref{fig:tela_cores}),

\begin{figure}[H]
    \centering
    \includegraphics{imagens/tela_cores.png}
    \caption{Blocos importantes de ações na tela inicial}
    \label{fig:tela_cores}
\end{figure}

\vspace{0.2cm}

sendo:
\begin{enumerate}
    \item Opções de conta e comunidade;
    \item \label{it:segundo} Ferramentas de criação de arquivos;
    \item Listagem de arquivos e protótipos criados;
    \item Opções de equipe. \\
\end{enumerate}
\vspace{0.2cm}

Cada ferramenta de criação possuem seu próprio layout, que serão explicados na \autoref{subsec:menus_elementos} abaixo.

\subsection{Menus, botões e elementos importantes}
\label{subsec:menus_elementos}

Ao escolher uma das Ferramentas de criação do \autoref{it:segundo}, é aberta uma nova janela com um novo layout  de acordo com a ferramenta escolhida, será apresentado cada uma destas opções abaixo:

% ----------------------------------

\subsubsection{Figma Design}
\label{subsubsec:menu_design}

\begin{figure}[H]
    \centering
    \includegraphics{imagens/tela_design.png}
    \caption{Blocos importantes de ações no Figma Design}
    \label{fig:tela_design}
\end{figure}

Neste layout, existem os seguintes blocos:
\begin{enumerate}
    \item Opções de informações do arquivo;
    \item Opções de camadas do arquivo;
    \item Tela principal para manipulação dos recursos de design;
    \item Ferramentas de criação de recursos gráficos;
    \item Opções de colaboração;
    \item Opções do objeto selecionado.
\end{enumerate}
\vspace{0.2cm}

% ----------------------------------

\subsubsection{FigJam}
\label{subsubsec:menu_figjam}

\begin{figure}[H]
    \centering
    \includegraphics{imagens/tela_jam.png}
    \caption{Blocos importantes de ações no FigJam}
    \label{fig:tela_jam}
\end{figure}

Neste layout, existem os seguintes blocos:
\begin{enumerate}
    \item Opções de informações do arquivo;
    \item Opções de colaboração;
    \item Tela principal para manipulação de objetos;
    \item Ferramenta de criação dos objetos.
\end{enumerate}
\vspace{0.2cm}

% ----------------------------------

\subsubsection{Figma Slides}
\label{subsubsec:menu_slides}

\begin{figure}[H]
    \centering
    \includegraphics{imagens/tela_slides.png}
    \caption{Blocos importantes de ações no Figma Slides}
    \label{fig:tela_slides}
\end{figure}

Neste layout, existem os seguintes blocos:
\begin{enumerate}
    \item Opções de informações do arquivo;
    \item Opções de manipulação de ordem e criação de slides;
    \item Tela principal para manipulação do slide escolhido;
    \item Ferramentas de criação de objetos para o slide;
    \item Opções de colaboração;
    \item Opções do objeto selecionado.
\end{enumerate}
\vspace{0.2cm}

% ----------------------------------

\subsubsection{Figma Buzz}
\label{subsubsec:menu_buzz}

\begin{figure}[H]
    \centering
    \includegraphics{imagens/tela_buzz.png}
    \caption{Blocos importantes de ações no Figma Buzz}
    \label{fig:tela_buzz}
\end{figure}

Neste layout, existem os seguintes blocos:
\begin{enumerate}
    \item Opções de informações do arquivo;
    \item Ferramentas de criação de recursos gráficos;
    \item Tela principal para manipulação dos recursos de design;
    \item Opções de troca de modos de uso (visualização, edição, comentários, desenvolvimento)
    \item Opções de colaboração;
\end{enumerate}
\vspace{0.2cm}

% ----------------------------------

\subsubsection{Figma Sites}
\label{subsubsec:menu_sites}

\begin{figure}[H]
    \centering
    \includegraphics{imagens/tela_sites.png}
    \caption{Blocos importantes de ações no Figma Slides}
    \label{fig:tela_sites}
\end{figure}

Neste layout, existem os seguintes blocos:
\begin{enumerate}
    \item Opções de informações e camadas do arquivo;
    \item Opções de configurações do arquivo (database, telas, pesquisa, etc.);
    \item Tela principal para manipulação dos recursos;
    \item Ferramentas de criação de recursos gráficos;
    \item Opções de colaboração;
    \item Opções do objeto selecionado.
\end{enumerate}
\vspace{0.2cm}

% ----------------------------------

\subsubsection{Figma Make}
\label{subsubsec:menu_make}

\begin{figure}[H]
    \centering
    \includegraphics{imagens/tela_make.png}
    \caption{Tela inicial do Figma Make}
    \label{fig:tela_make}
\end{figure}

É preciso primeiramente enviar o que será gerado ao Figma, após enviar as especificações, abre-se uma janela com um layout similar aos demais:

\begin{figure}[H]
    \centering
    \includegraphics{imagens/tela_make_blocos.png}
    \caption{Blocos importantes na tela do Figma Make}
    \label{fig:tela_make_blocos}
\end{figure}

Neste layout, existem os seguintes blocos:
\begin{enumerate}
    \item \href{https://pt.wikipedia.org/wiki/Chatbot}{Chatbot interativo};
    \item Opções de visualização;
    \item Tela principal para visualiação do objeto gerado;
    \item Opções de colaboração.
\end{enumerate}
\vspace{0.2cm}


% ----------------------------------------------------------------------------------------

% FUNCIONALIDADES %

% ----------------------------------------------------------------------------------------

\section{Principais Funcionalidades}
\label{sec:funcionalidades}

Após apresentar as principais telas do sistema, é preciso entender as suas principais funcionalidades. Por ser um sistema com diversas funcionalidades para suas várias ferramentas, será apresentado as principais funcionalidades da ferramenta mais desenvonvida e utilizada no Figma, o Figma Design (layout pode ser visto na \autoref{fig:tela_design})

% ----------------------------------

\subsection{Criar um novo arquivo de design}
\label{subsec:criar_design}

\textbf{Passo a passo}
\begin{enumerate}
    \item Na tela inicial, clique em \textit{Design} (Segundo botão, da esquerda para a direita, do bloco 2 da \autoref{fig:tela_cores});
    \item Aguarde o carregamento do editor;
    \item Renomeie o arquivo clicando no título no topo. (Bloco 1 na \autoref{fig:tela_design}).
\end{enumerate}

% ----------------------------------

\subsection{Criar um novo frame}
\label{subsec:criar_frame}

\textbf{Passo a passo}
\begin{enumerate}
    \item Clique no ícone de \textit{Frame} (Segundo botão da esquerda para a direita da \autoref{fig:tela_design}, localizado no bloco 4);
    \item Escolha um tamanho predefinido no bloco 6 (por exemplo, iPhone 16);
    \item Irá ser criado um frame que será utilizado para a prototipação, todos os objetos em um mesmo frame aparecerão em uma mesma tela (\autoref{fig:tela_design_blocos}).
\end{enumerate}

\begin{figure}[H]
    \centering
    \includegraphics{imagens/tela_design_frame.png}
    \caption{Criação de um frame}
    \label{fig:tela_design_blocos}
\end{figure}

% ----------------------------------

\subsection{Prototipar interações}
\label{subsec:criar_interacoes}

\textbf{Passo a passo}
\begin{enumerate}
    \item Abra a aba \textit{Protótipo} (visto no Bloco 6 da \autoref{fig:tela_design});
    \item Clique no botão interações;
    \item Selecione qual o gatilho da interação, sua ação e destino;
    \item Defina a animação e o comportamento no painel direito.
\end{enumerate}

\begin{figure}[H]
    \centering
    \includegraphics{imagens/tela_design_prototipo.png}
    \caption{Etapa de prototipação de interação entre telas}
    \label{fig:tela_design_prototipo}
\end{figure}

% ----------------------------------

\subsection{Adicionar um objeto no frame}
\label{subsec:criar_objeto}

\textbf{Passo a passo}
\begin{enumerate}
    \item Clique no ícone de objeto (Terceiro botão da esquerda para a direita da \autoref{fig:tela_design}, localizado no bloco 4);
    \item Escolha um objeto para criá-lo (por exemplo, retângulo);
    \item Irá ser criado um objeto, que pode ter suas propriedades editadas posteriormente (\autoref{fig:tela_design_objeto}).
\end{enumerate}

\begin{figure}[H]
    \centering
    \includegraphics{imagens/tela_design_objeto.png}
    \caption{Criação de objeto}
    \label{fig:tela_design_objeto}
\end{figure}

% ----------------------------------

\subsection{Editar um objeto no arquivo}
\label{subsec:editar_objeto}

\textbf{Passo a passo}
\begin{enumerate}
    \item Clique objeto desejado;
    \item Escolha as propriedades que deseja modificar (Bloco 6 na \autoref{fig:tela_design});
    \item Ao apertar \textit{enter}, o objeto será modificado.
\end{enumerate}

\begin{figure}[H]
    \centering
    \includegraphics{imagens/tela_design_editar.png}
    \caption{Edição de objeto}
    \label{fig:tela_design_editar}
\end{figure}

% ----------------------------------

\subsection{Comentar em algum objeto}
\label{subsec:criar_comentário}

\textbf{Passo a passo}
\begin{enumerate}
    \item Clique no ícone de comentário (Sexto botão da esquerda para a direita da \autoref{fig:tela_design}, localizado no bloco 4);
    \item Escolha um objeto comentar;
    \item Irá ser criado um comentário, visível para todos que tem acesso ao arquivo (\autoref{fig:tela_design_comentario})
\end{enumerate}

\begin{figure}[H]
    \centering
    \includegraphics{imagens/tela_design_comentario.png}
    \caption{Criação de comentário}
    \label{fig:tela_design_comentario}
\end{figure}

% ----------------------------------

\subsection{Desenhar no arquivo}
\label{subsec:criar_desenho}

\textbf{Passo a passo}
\begin{enumerate}
    \item Clique no ícone de comentário (Sexto botão da esquerda para a direita da \autoref{fig:tela_design}, localizado no bloco 4);
    \item Escolha a ferramenta de desenho;
    \item Desenhe o que desejar no \textit{canvas}(\autoref{fig:tela_design_desenho}).
\end{enumerate}

\begin{figure}[H]
    \centering
    \includegraphics{imagens/tela_design_desenho.png}
    \caption{Criação de desenho}
    \label{fig:tela_design_desenho}
\end{figure}

% ----------------------------------

\subsection{Compartilhar o link para visualização}
\label{subsec:compartilhar_design}

\textbf{Passo a passo}
\begin{enumerate}
    \item Clique no ícone Compartilhar (Terceiro botão da esquerda para a direita da \autoref{fig:tela_design}, localizado no bloco 5);
    \item Escolha se deseja compartilhar por e-mail ou link (\autoref{fig:tela_design_compartilhar}).
\end{enumerate}

\begin{figure}[H]
    \centering
    \includegraphics{imagens/tela_design_compartilhar.png}
    \caption{Menu de compartilhamento}
    \label{fig:tela_design_compartilhar}
\end{figure}

% ----------------------------------

\subsection{Mover o arquivo para uma equipe}
\label{subsec:mover_design}

\textbf{Passo a passo}
\begin{enumerate}
    \item No menu de compartilhamento (\autoref{fig:tela_design_compartilhar}), clique em Mover Arquivo;
    \item Escolha qual equipe você deseja mover o arquivo (\autoref{fig:tela_design_mover}) e clique em "mover";
    \item O arquivo será movido para a pasta da equipe e todos os participantes podem colaborar nele.
\end{enumerate}

\begin{figure}[H]
    \centering
    \includegraphics{imagens/tela_design_mover.png}
    \caption{Escolha de equipe}
    \label{fig:tela_design_mover}
\end{figure}

% ----------------------------------

\subsection{Deletar o arquivo}
\label{subsec:deletar_design}

\textbf{Passo a passo}
\begin{enumerate}
    \item Na tela inicial (\autoref{fig:tela_principal}), selecione o arquivo que deseja apagar;
    \item Clique com o botão direito do mouse e escolha a opção "mover para a lixeira" e confirme sua ação;
    \item O arquivo será movido para a lixeira da equipe que foi criada. Para acessá-la, aperte em "lixeira" (Quinto botão de cima para baixo do bloco 4, visto na \autoref{fig:tela_cores})
    \item Irá ser criado um objeto, que pode ter suas propriedades editadas posteriormente (\autoref{fig:tela_design_objeto})
    \item Ao localizar o arquivo na lixeira, clique com o botão direito do mouse e escolha a opção "excluir permanentemente";
    \item O arquivo será exlcluído permanentemente e não pode ser recuperado.
\end{enumerate}

\begin{figure}[H]
    \centering
    \includegraphics{imagens/tela_design_excluir_permanentemente.png}
    \caption{Opções na lixeira}
\label{fig:tela_design_excluir_permanentemente}
\end{figure}

% ----------------------------------

\end{document}
